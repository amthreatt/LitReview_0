\documentclass[10pt,twocolumn]{article} 

\usepackage{oxycomps} % use the main oxycomps style file


\bibliography{references}

\pdfinfo{
    /Title (COMPS Proposal)
    /Author (Justin Li)
}

\title{Oxy Comps Paper Part 1 \LaTeX}

\author{Amelia Threatt}
\affiliation{Occidental College}
\email{athreatt@oxy.edu}

\begin{document}

\maketitle
\section{Introduction}
For the Comprehensive requirement for the Computer Science Major, I have decided to construct my own automated online math tutoring machine. The math tutoring machine will be a website designed to take in single variable algebraic equations and provide the user with guided prompts to help students solve for a variable, while simultaneously checking their work at each step. Assistive technology (AT) has been used to help students with learning disabilities, by providing a helpful learning tool that can be utilized in the classroom and out of the classroom. The difference between my website and typical assistive technology is my website will be designed for anyone who needs additional math help whereas assistive technology is marketed towards and designed specifically for students with learning disabilities. 


\section{Problem Context}
Every year California Public School and Charter School students, third to eighth grade and eleventh grade, take a standardized test. This test covers English language arts and mathematics. Of those two subjects, they get broken down into a few categories to create a well-rounded exam to assess how well the students are learning new material accurately. These test results have continually shown that students perform better in English language arts instead of Mathematics. This noticeable trend brought common core math teaching to California schools; however, this disparity in test results still stands. \cite{noauthor_california_nodate}
 
There are several reasons why a student may struggle with math; many other learning differences can attribute struggles with math, such as dyscalculia, dysgraphia, dyslexia, visual processing disorders, or add/ADHD much more that were not mentioned \cite{noauthor_why_nodate}. Along with those previously mentioned learning differences, math anxiety is an issue some students can suffer from without having a learning difference attached to it. Essentially everyone who struggles with math struggles differently from the next person. However, there are identifiable fundamentals where the challenges lie.  
 
Many different teaching techniques have been designed to best help students in and outside of class with the education of math and other subjects. One way of teaching is a sort of the modified Orton Gillingham (OG)methodology adapted for mathematics teaching and a focus of evidence-based instruction. Orton and Gillingham were researchers who focused on reading struggles, and language processing disorders, along with dyslexia, they together created a set of teaching principles. The Orton-Gillingham approach is a “direct, explicit, multisensory, structured, sequential, diagnostic, and perspective way to teach literacy when reading, writing, and spelling” \cite{ahearn_what_2016}. The Orton-Gillingham method was designed for reading and writing; however, educators have begun implementing this approach into maths teaching. In math teaching the methodology starts by creating concrete examples for students so creating more hands-on examples when teaching, and then finally converting what they have learned into math symbols and numbers. Essentially the OG approach in math does use the multisensory aspect with the concrete examples, but along with that the teaching is data-driven, it will implement direct connections between previous material and new material, and immediate feedback \cite{noauthor_why_nodate}. Along with the OG methodology, there is evidence-based math teaching that has been helping develop the way math is taught. Evidence-based math has four main focuses: explicit instruction, visual representation, schema-based instruction, and peer interaction \cite{noauthor_evidence-based_nodate}. The aspect of evidence-based teaching that can be applied to virtual tutoring will be explicit instruction which is a way of making the learning process very clear; this helps students not have to guess what to do as often. Breaking down a problem into clear and identifiable steps can be helpful for students who cannot retain everything at once. As well as the evidence-based instruction a website provides the option to also implement visual representations as well. 
Online tutoring is not new, especially due to the pandemic there has been a lot of services that have transferred online. With most typical online tutoring services, there is a cost attached, but with that being said you are provided with live one on one tutoring. However, the price of these services recessed the accessibility to math education aids, although a website is not a person have access to a free and comprehensive response to a student input can be helpful when a one on one tutor is not an option. 
 
With all that said, the main points of math education I want to bring to this website are structured and sequential aspects from OG, along with the immediate feedback, and having clear and identifiable prompts for the student to help guide to the next step in an algebraic problem. But most importantly what I have taken from my research will be the visual aspects of the website. Typing math symbols into a computer have not always been so easy, but websites like Wolfram alpha have made great efforts to make mathematical user input much easier. 




\section{Technical Background}
The main algorithmic details of this website revolve around interpreting and organizing the user input, and then with that understanding how to use the information to best help the student. Wolfram Alpha for example is a powerful online calculator that with a premium subscription provides step by step details for a problem. To provide the step by step information as well as preform the computations the website uses its own programming language in combination with Mathematica. However this is not the approach all online calculators use, and is not the approach that will be used for this new website. Due to the capacity of the website the user is confined to only inputting quadratic, or just simple algebra problems. With that being said the step by stem functionality of the website can be implemented by converting the equation to a tree data structure.   

In the wolfram programming cloud, it uses its own language; however, it gives access to a number of mathematical tools that will be helpful in the creating of this website. The wolfram language is said to be the only full scale computational language; the language already has a number of complex algorithms worked into it, making high level computation much easier. Although the website planned does not require high level computation, it will require a lot of small computations after every input. 

However, the complexity of the calculation performed will not be extensive, I have decided to set the focus of this website on being able to take in any quadratic equation. The website will create specialized steps, prompts to guide the student and real-time checking of their inputs as they work towards the solution. With that being said, this website will be built using JavaScript. The construction of this website will be inspired by an online integral calculator website https://www.integral-calculator.com/. So similar to this website I will use JavaScript to create a parser to take the user input and configure it into a tree to then use to guide the students work with step by step prompts. With each prompt it will take in a user input, I will also be checking that input to make sure the student performed the math correctly as this step. To display the information and perform the computations I will use Yet Another Computer Algebra System (YACAS). 

\section{Prior Work}
Many online calculators provide step by step solutions to a problem however a difference between mine and other websites is my step by step solution will be user-inspired. Essentially the user will be writing out the step by step solution themselves. The new website however will be checking to see if the steps attempted are correct, along with that if a step is incorrect then the website will also provide guiding prompts to help get the student back on track towards the correct solution. 

With that being said, the other online calculators are helpful in showing what are helpful design elements and approaches towards creating this website. For example, Wolfram Alpha is not only capable of complicated computations it also has a clean design and an understandable interface. Another important online calculator that has been a big inspiration for my website is https://www.integral-calculator.com/. 

Aside from online calculators, there exists a Math Robot tutor that does provide real-time feedback for the student as well as providing why an answer is correct or not. This math robot is called ABii, along with the fully autonomous physical robot body it also has a comprehensive online math teaching program along side with it. A qualm with the ABii is  the price tag, the price of the ABii k-5 School Edition+ is $1499.00$. Although this robot can be a very useful teaching tool, this price makes it inaccessible to lower-income students or schools. 

Even with Wolfram Alpha to get access to the premium website has a monthly cost. Adding fees to these websites and tools automatically reduces the accessibility to education tools. As mentioned, math education is still developing. The implementation of new math tools can help with that development, but realistically only if those tools are accessible to all types of students. 

Now as for automated online math tutoring machines, I have found one that uses machine learning to really lean into the personalized teaching information. The website will attempt to become an intelligent tutoring system. An intelligent tutoring system have been made before, the definition for an intelligent tutoring system is a computer system that provides personalized and immediate feedback for a student.  

\section{Ethical Consideration}
 In the construction of an automated online math tutoring machine, there are ways to address some ethical concerns however it is not possible to provide a solution to all concerns on the one website. This is because when addressing one learning disability the implications could be negative for a different learning disability. Other than learning disabilities educational websites are not optimal for accessibility, given the scope of this website only being able to solve algebraic problems this tool will concentrate the power to a small group of math students. 

Intelligent tutoring systems can be helpful in increasing the accessibility of tutoring to more people, especially if the website is free which mine will be. The website being free will remove any financial burden that may come with hiring a tutor. Utilizing the web platform will also allow the student to access the information and help at any time, and from anywhere which may be helpful for students with mobility issues. Although these systems can be helpful to some it does come with some pitfalls of being unable to provide everyone with everything they need for the website to be most successful. 

There are instances where an online intelligent tutoring system is not going to be beneficial for some students, due to various accessibility issues, cognitive and learning disorders, and physical impairments. At this point and time not everyone has access to the internet, and even if they do not everyone is knowledgeable about the internet and how some devices work. Websites inherently are not completely accessible.

Due to the lack of accessibility within the internet, the website will be monopolizing power to people who have access to the internet which is not an equally diverse group of people. Along with that the website will be most easily accessible to those with computer literacy and are able to navigate the website from a compatible device. 

In an attempt to create a fully ethical website the I should take into account all forms of where accessibility will be challenged and create a solution, however, a solution for one problem could create an entirely new problem for another people. To an extent the accessibility features can be user-customized but at what point will all forms of inaccessibility be challenged begging the question of technological solutionism. At the end of the day, in-person tutors came first, intelligent tutoring systems came afterward, if there a way to design a website full capable of the personalization an in-person tutor could provide?

Aside from the ethics behind the purpose of the website there are other aspects of ethical considerations to take into account, such as if websites can be ethically made. Well there have been papers and books published on the matter. The web development community is seen making efforts to identify where web apps could be ethically improved. The main places where my web app could be improved is in the accessibility aspect, in regards to addressing the needs of various disabilities, and then also will internet accessibility for young students. The target audience of the website is an age group who doesn't have the ability to provide their own technology, so they depend on their guardians and the schools to provide them. However, there are socioeconomic factors at play that do not make this an accessible resource for all students. Not being able to guarantee lower-income students access to this website simply due to the fact that they do not have access to technology is only aiding more in the power divide between them and other students who d have the ability to access the internet. Yet the only way to assuage the inequity would be to regulate who gets to use the website and who does not, and arguably there is no ethical way to choose which students should get access to a free public website ethically.  


\section{Methods and Evaluation}
When creating an education tool it is important to find a useful design, to do this it is helpful to receive feedback with each step of creating the website. User interface and experience are pivotal for creating a engaging yet non distracting website for student users. I will be implementing the ADDIE Model which stands for Analysis Design Development Implementation Evaluation model. ADDIE is one of the more popular Instructional System Design models which is used to create a useful instructional process that ensures adequate performance by students. Addie essential relates each step in the process back to evaluation to guarantee the success in each individual element.\cite{borrelli_user_nodate}

So the first part of ADDIE is analyze, throughout this paper I have analyzed the needs and recommended process to create an education tool. I have found the data that supports my stance of middle school students struggling with math, and in the ethics component of the paper I go over how an online teaching tool is one of the more accessible aids for most students. However to best evaluate my analysis I would find a middle school teacher and ask their thoughts on assistive technology. I would also like their input on the Orton-Gillingham model of teaching and its applications to mathematics for students with and without disabilities and if this model can be equally useful both types of students. 

Design is going to be one of the most important steps of the ADDIE process, once a design is settled on and the development begins then there is not a lot of modifications that can be made without starting the development all over again. So to ensure my design is most effective as possible I will be receiving feedback from middle school teachers, and even middle schools on elements they would like to see in the website, and elements that should not be included to reduce distraction while still being engaging. The design of my website will take inspiration from already existing online teaching tools such as Khan Academy, or Wolfram Alpha. Given that this is a math education tool I will ensure the numbers and formatting is clear and understandable for a middle school student. I think while designing the website I will have to remember to not make assumptions about the user, for example with multiplication there are many different ways to represent it however I cannot assume my user will be familiar with the way I choose to represent it on the website. A preamble of the website or a sort of guide page will be essential in making sure all people of all mathematics backgrounds can enjoy the website. 

This website will not be storing any user data so it will focus mainly on front end development. My plans for the computations aspect on my end to verify a students work is to use YACAS. The Computer Algebra System will really only be able to do math, so it is my job to use YACAS to solve an algebra problem the same way a student would set by step. Then I need to check if the student and I are at the same step if not then I need to find where the student's answer differs from the systems answer at that step and then provide guiding feedback on how to get the student to the same place as the system before moving on wards. There has been previously implementations of step by step calculators by using machine learning however that will not be the approach I take. What I will do is create a hierarchical ranking of operations and use a sort of reinforcement learning approach to teaching the website how to solve a problem. Once the system has an understanding then with each step it will compare with the user for validation. 

Along with utilizing YACAS to create the step by step guidance I will need to use to trees to store each side of the equation. After one mathematical step is done to one side of the equation then I will add it to the other side of the equation. This process will repeat until one tree consists only of the variable from the algebra problem and the other side of the tree consists of the operations required to get the solutions. To create these trees I can use the shunting yard algorithms that stores each number as a child and each node is an operation. 

The implementation aspect of the ADDIE model works to find the best way to integrate the online tool into the students life may it be in the classroom, or perhaps more of an at home assistive tool. This would require teacher feedback and their thoughts on online tools used during class. Often times students can get distracted on the internet so I would find this website to be more useful as like an additional tool for helping with homework. This website really is for students who desire a sort of hand holding through a math equation. As mentioned before math anxiety is a common issue many people struggle with causing math to be a more difficult subject for them. So a website like this is for people who develop frustration from solving math problems. Rather than telling a user they are wrong after spending all this time on a math problem, the website catches the mistake right away making it easier for the student to have a desire to carry on throughout a math problem. 

The last part of ADDIE is evaluation, and although I would have been evaluating at each steps of the method I would now have to evaluate the overall functionality of the website. To evaluate the website I will need to check for three things, does the website work, is it usable and does the website succeed in its goal. The website works if it steers the student in the direction of the correct answer as opposed to simply proving the step by step. In doing so the website should be able to identify where a student made a mistake and then provide help getting it to the correct answer. To test this aspect I will use curate a set of commonly made errors in algebra to see if the website catches them and then properly addresses them. 

Then to test the usefulness of the website among students I would offer the same quality of testing at 2 stages. So I would provide a group of students with a set of algebra problems and ask them to solve them in an open note manner. During this initial testing I will also be timing them to see how long it takes for them to complete the problem set. The I would ask them to complete another algebra problem set but only using my website as their resource, and time them as well. This step of evaluation which be able to show the efficiency of my website. Then to check the actually tutoring capabilities of the website I would like to preform a more long term testing. Where I would first need a control group I would provide them with a problem set, and along with them I will be giving another group of students the same problem set. Then after a month I will give the control group and the testing group another tests of the same level of difficulty a month later. I will use this change in score to determine if my website will actually make any significant difference among the control group and the students who got to use the website for a month with mathematics help. %The Califorina standardized testing has a lot of this information 


Next to see if the website is usable there are some key functionality items that need to be addressed. Such as can the website take in a mathematical input? Next can it recognize and problem it can solve, or a problem it cannot solve? Lastly does it provide the helpful guidance and is the math on the website readable? To check if my website can take in a mathematical problem I will be test inputting problems I know it can solve and problems I know it cannot solve. It should provide a response to the problems it cannot solve the student knows that this website unfortunately cannot help them with their problem. Testing the usefulness of the websites guiding questions will be tested more so in the question of did the app succeed in its goal. After providing the students with problem set to not only test their knowledge but also test the efficacy of the website I will also provide a set of qualitative questions to gauge the usefulness of the website for a student.

So far the ADDIE model has been useful in mainly technical evaluation, there are however multiple aspects of user evaluation to take into account. Such as if Orton Gillingham model actually useful in my website creation and its applications to mathematics. Along with that this website was designed to be useful for students however it would be valuable to evaluate its usefulness among teachers. Such as if this is a website teachers would want their students using, and if a website teaching tool is something that can be useful in class.

\section{Timeline}

\textbf{4/15:} 
Complete Ethics Paper

\textbf{5/1:} 
Proposal and Tutorial completed

\textbf{5/15:}  
Start working on goal of website, practice implementing algorithms mentioned in methods. At this point I should start thinking about some of the design elements I want to go into my sketch of what I want the website should look like. 

\textbf{6/1:} 
By now I should have met with some teachers and gotten input on my design so that I can actually begin some of the development. I would first like to start by making a Step by Step calculator, so that I can implement it into the website.

\textbf{6/15:}  
Now I should have the website working for the most part and the rest of this time will be spent fine tuning the display, and working on integrating an instruction page to help the user understand how to use this website. The sooner the website is working the more time I will get to add in additional accessibility features, such as a color changing more for students with forms of color blindness.

\textbf{7/1:} I should have a rough mock up of website with appropriate inputs and outputs. It does not have to be working perfectly however I should be able to test what inputting function looks like and if my code is providing accurate feedback


\textbf{11/15:} 
Poster is due 

\textbf{12/5:} 
Poster Presentation

\textbf{12/15:} 
Final paper and code due.


\begin{comment}
No definition citations, unless the term itself is in dispute
Separate problem background from technical background
    Unclear if games and apps require much technical background
    The general structure of the framework might be better suited for the Architecture Overview section
        Eg. Flask uses decorators to associate functions with URLs
        Eg. Unity has scripts associated with objects and specific triggers, such as walking into an area, pressing a button, etc.
    Maybe a better name is "algorithmic background"?
        Should explore what does and doesn't count
            All ML counts
            App and game frameworks do not
        Framework vs. library?
            I like the idea of [inversion of control](https://martinfowler.com/bliki/InversionOfControl.html), but that may be too abstract for students to understand
        Heuristic: is understanding that system necessary to understand the results?
            Ie. How Flask or Unity works doesn't influence whether the app/game is useful/fun/engaging
            But how (say) linear regression works is highly relevant for why the results match/don't match the actual values
\end{comment}

\printbibliography 

\end{document}
